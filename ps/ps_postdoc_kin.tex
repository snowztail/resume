\documentclass[10pt]{scrartcl}

\usepackage{scrletter}
\usepackage{parskip}

\usepackage{cite}
\usepackage{import}
\usepackage{microtype}
\usepackage[colorlinks]{hyperref}
\usepackage[acronym]{glossaries-extra}
\usepackage[subtle]{savetrees}

% letter signature
% \renewcommand*{\raggedsignature}{\raggedright}

% \setlength\parindent{0pt}

% glossaries-extra
\glsdisablehyper
\setabbreviationstyle[acronym]{long-short}
\newacronym{itp}{ITP}{Intelligent Transmission and Processing}


\begin{document}

\setkomavar{date}{\today}
\setkomavar{signature}{Yang Zhao}
\setkomavar{subject}{Postdoctoral Application -- Research Fellow in Intelligent Communications \href{https://www.imperial.ac.uk/jobs/description/ENG02922/research-associate-intelligent-communications/}{(ENG02922)}}

\begin{letter}{%
		Prof Kin K. Leung and Dr Stefan Vlaski\\
		Communications and Signal Processing Group\\
		Department of Electrical and Electronic Engineering\\
		Imperial College London
	}
	\opening{Dear Prof Leung and Dr Vlaski,}

	Hope this letter finds you well.

	I am writing to express my enthusiastic interest in the Research Fellow path 3 (AI and ML for physical layer).

	I have been working on physical layer wireless at Imperial with \href{https://www.imperial.ac.uk/people/b.clerckx}{Prof Bruno Clerckx} for 5 years.
	Topics covered include MIMO, reconfigurable intelligent surface (RIS), backscatter communications (BackCom), and simultaneous information and power transfer (SWIPT).
	I also read intensively on rate splitting multiple access (RSMA), index modulation (IM), and symbiotic radio (SR).
	\href{https://scholar.google.co.uk/citations?user=ckmF3VsAAAAJ&hl=en}{My publications} include 1 journal on \href{https://github.com/snowztail/irs-aided-swipt-joint-waveform-active-and-passive-beamforming-design-under-nonlinear-harvester-model}{RIS x SWIPT}, 1 journal on \href{https://github.com/snowztail/riscatter-unifying-backscatter-communication-and-reconfigurable-intelligent-surface}{RIScatter}, and 1 RIS book chapter.
	I am currently working on MIMO channel shaping and interference alignment using beyond diagonal RIS, where another 2 journal papers are close to submission.
	There is another half-cooked manuscript on RSMA x IM where the antenna index conveys common information.
	The number of publications is not impressive, but I really enjoy the explorations and am proud of the quality and impact.

	The position is very appealing as I am looking forward to new challenges and perspectives.
	So far I have been following a theoretic path of signal processing, optimization, information theory, and some learning basics.
	Those are extremely useful tools for beginners to understand the limits and essence of physical layer.
	However, the society are very close to the Shannon limit and many papers still abuse those tools for too-ideal or too-complicated problems.
	One day I came across the \href{https://en.wikipedia.org/wiki/Four_color_theorem}{Four Color Problem} that was unpuzzled by the world's first machine-aided proof.
	This shook the math society 50 years ago, but now Lean has become a standard tool for mathematicians to test and inspire proofs.
	I feel the same story is happening in the physical layer --- decades ago people talked about ``genie-aided'' approaches in error correction and cognitive radio, and today we can indeed mimic its power by AI and ML.
	I am exciting to be a weak learner that contributes to the model of future communications.

	It would be a unique opportunity to work with another world-leading team at CSP group and shift my focus to learning-empowered communications.
	One of my disadvantage is that I have not authored any learning-related papers.
	However, I have finished the machine learning (EE3-23), deep learning (EE3-25), pattern recognition (EE4-68), computer vision (EE4-62), adaptive signal processing and machine learning (EE4-13) at Imperial, and have been taking IEEE training courses as well.
	On the other hand, I have been working on the cutting-edge topics and my background in communication and optimization might be a good fit for the position.
	I am confident to quickly adapt to the new environment and make fresh contributions, and I am looking forward to working with you.

	Sincerely appreciate your time and consideration. Wish you can find the perfect candidates.
	\closing{Yours sincerely,}
\end{letter}


\end{document}
