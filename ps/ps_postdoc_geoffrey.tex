\documentclass[10pt]{scrartcl}

\usepackage{scrletter}
\usepackage{parskip}

\usepackage{cite}
\usepackage{hyperref}
\usepackage{import}
\usepackage{microtype}
\usepackage[acronym]{glossaries-extra}
\usepackage[subtle]{savetrees}

% letter signature
% \renewcommand*{\raggedsignature}{\raggedright}

% \setlength\parindent{0pt}

% glossaries-extra
\glsdisablehyper
\setabbreviationstyle[acronym]{long-short}
\newacronym{itp}{ITP}{Intelligent Transmission and Processing}


\begin{document}

\setkomavar{date}{\today}
\setkomavar{signature}{Yang Zhao}
\setkomavar{subject}{Postdoctoral Application -- Research Associate in ITP Lab \href{https://www.imperial.ac.uk/jobs/description/ENG02797/copy-research-assistant-intelligent-transmission-and-processing/}{(ENG02797)}}

\begin{letter}{%
		Professor Geoffrey Ye Li\\
		Chair Professor of Wireless Systems
	}
	\opening{Dear Prof Li,}
	I am writing to express my enthusiastic interest in the Research Associate position that aims at bringing the state-of-the-art learning techniques to wireless communications and signal processing.

	I am currently a PhD candidate in wireless at Imperial College London, supervised by Prof Bruno Clerckx and to graduate by May 2024.
	My current research focuses on MIMO, reconfigurable intelligent surfaces, backscatter communications, and wireless power transfer.
	I have \href{https://scholar.google.co.uk/citations?user=ckmF3VsAAAAJ&hl=en}{published} a book chapter and 2 top-tier transactions papers with 2 more close to submission.

	The reason I am prioritizing this particular application is two-fold.
	First, I have been interested and trained in statistics and learning for years, but have not had the chance to engage the latest research in learning x communications.
	During my PhD I have been following a conventional path over optimization, analysis, and information theory, which are extremely useful tools for beginners to understand the essence and fundamental limits.
	Recently I feel the urge for a fresh perspective, as more papers are over-relying on these tools to solve piled-up problems (after carefully reviewing some make little sense!)
	One day I came across the \href{https://en.wikipedia.org/wiki/Four_color_theorem}{Four Color Problem} that was answered by the world's first machine-aided proof.
	It shook the society 50 years ago.
	Now Lean has become a standard tool for mathematicians to prove and test statements.
	Mark Twain once said ``History never repeats itself, but it does often rhyme.''
	It feels to me that the rhyme has come to deep learning, and I may just become a weak learner that contributes to the model of future communications.

	Second, I have been following Prof Li's publication -- there are always revolutionary works from time to time.
	Once I had the opportunity to present the impressive idea of deep learning-enabled semantic communication in my research group.
	Everyone was super interested and the slides were well-received, but I felt a bit embarrassed because it was not my work.
	Also, the ground-breaking symbiotic radio works profoundly inspired my proposal of RIScatter, and Prof Li gave me so much valuable feedback that I could not be more grateful.
	Today I am still benefiting from the philosophy ``Less words, more meaning.''
	It would be an exciting opportunity to work with the world-leading communication lab and shift my focus to learning-empowered communications.

	One of my disadvantage is that I have not authored any learning-related papers.
	However, I have finished the machine learning (EE3-23), deep learning (EE3-25), pattern recognition (EE4-68), computer vision (EE4-62), adaptive signal processing and machine learning (EE4-13) at Imperial, and have been taking IEEE training courses as well as reading hot papers.
	On the other hand, I have been working on the cutting-edge topics and my background in wireless communications, information theory, optimization, and signal processing might benefit the lab in terms of diversity.
	I am confident to quickly adapt to the new environment and become a good fit for the position, and I am looking forward to working with you again.

	Thank you for your time and consideration. Wish you a better year 2024.
	\closing{Yours sincerely,}
\end{letter}


\end{document}
