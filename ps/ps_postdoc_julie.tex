\documentclass[10pt]{scrartcl}

\usepackage{scrletter}
\usepackage{parskip}

\usepackage{cite}
\usepackage{import}
\usepackage{microtype}
\usepackage[colorlinks]{hyperref}
\usepackage[acronym]{glossaries-extra}
\usepackage[subtle]{savetrees}

% letter signature
% \renewcommand*{\raggedsignature}{\raggedright}

% \setlength\parindent{0pt}

% glossaries-extra
\glsdisablehyper
\setabbreviationstyle[acronym]{long-short}
\newacronym{itp}{ITP}{Intelligent Transmission and Processing}


\begin{document}

\setkomavar{date}{\today}
\setkomavar{signature}{Yang Zhao}
\setkomavar{subject}{Postdoctoral Application -- Research Fellow in Integrated Sensing and Communications \href{https://www.imperial.ac.uk/jobs/description/ENG02921/research-fellow-integrated-sensing-and-communications/}{(ENG02921)}}

\begin{letter}{%
		Professor Julie McCann\\
		Vice-Dean (Research) for the Faculty of Engineering
	}
	\opening{Dear Prof McCann,}

	Hope this letter finds you well.

	I am writing to express my enthusiastic interest in the Research Fellow position that aims at implementing integrated sensing and communications (ISAC) for CHEDDAR.

	I have been working on physical layer wireless at Imperial with \href{https://www.imperial.ac.uk/people/b.clerckx}{Prof Bruno Clerckx} for 5 years.
	Topics covered include MIMO, reconfigurable intelligent surface (RIS), backscatter communications, and simultaneous information and power transfer (SWIPT).
	I was lucky to \href{https://scholar.google.co.uk/citations?user=ckmF3VsAAAAJ&hl=en}{have published} 2 top-tier journal papers and a book chapter.
	By the end of my PhD (March 2024), 2 more high-quality journal papers will be submitted and another 1 will be half cooked.
	Most of my works are on RIS, a 6G candidate that redistributes ambient signals for improved wireless performance.

	The position is appealing to me for two reasons.
	First, sensing and communications are two sides of the same coin --- they are both about extracting information from the radio, one related to the environment and the other related to the people.
	The boundary is getting blurred during the paradigm shift from connectivity to intelligence that we are experiencing now.
	Sensing fits perfectly in my research vision of making the most of radio waves.
	After working on backscatter (radiates on demand), RIS (recycles signals on the air) and SWIPT (unifies information and energy delivery), I am ready for ISAC that completes the picture of intelligent wireless future, where everything and everyone understands and connects with the others better.

	Second, I am looking forward to new challenges and perspectives.
	So far I have been following a theoretic path of signal processing, optimization, information theory, and some machine learning.
	Those are extremely useful tools for beginners to understand the limits and essence.
	However, the society are very close to the Shannon limit, and many papers still abuse those tools for too-complicated or too-ideal problems.
	When discussing with colleagues, I often feel many of them care more about the publication (quantity and who first) than research quality and real-world impact.
	Instead, I really want to rethink from a fresh perspective and to solve practical problems.
	One day I came across the \href{https://en.wikipedia.org/wiki/Four_color_theorem}{Four Color Problem} that was unpuzzled by the world's first machine-aided proof.
	This shook the math society 50 years ago, but now Lean has become a standard tool for mathematicians to test and inspire proofs.
	It suggests that diversity is helpful not only in wireless channels, but also in research philosophy and life decisions.

	The AESE lab covers a wide range of topics.
	Of my particular interest are the works on LoRa and environment monitoring where protocols and prototypes have been developed.
	I read extensively on ambient backscatter and symbiotic radios but have not had the chance to implement them.
	Please allow me to share our theoretic work on \href{https://github.com/snowztail/riscatter-unifying-backscatter-communication-and-reconfigurable-intelligent-surface/blob/master/poster/beamer.pdf}{RIScatter} --- a batteryless cognitive radio that partially modulates the tag information and partially engineers the wireless channel by adaptive input distribution design.
	This idea of ``passive tags and sensors can talk over and enhance Wi-Fi'' has attracted much attention at a symposium, but we didn't have the resource to continue the project (Prof Clerckx wanted me to prioritize the other funded project).
	It would be exciting if we could try this at the AESE lab and demonstrate it in practice.

	One of my disadvantage is that I didn't have any experience on sensing.
	I am very happy to learn from the textbook, papers and colleagues.
	On the other hand, I have been working on the cutting-edge wireless topics and am confident in wireless communications, information theory, signal processing, optimization, and matrix analysis.
	I also helped as a teaching assistant for undergraduate and postgraduate communication courses for many years.
	Hopefully my background might benefit the lab in terms of diversity.

	Thank you so much for your time and consideration. Wish you a better year 2024.
	\closing{Yours sincerely,}
\end{letter}


\end{document}
